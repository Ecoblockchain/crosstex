\documentclass{article}
\usepackage{hyperref}

%HEVEA \renewcommand{\TeX}{TeX}
\newcommand{\XTeX}{Cross\TeX}
\newcommand{\BibTeX}{\textsc{Bib}\TeX}
%HEVEA \renewcommand{\par}{\begin{rawhtml}<p>\end{rawhtml}}

\title{\XTeX{} Tutorial}
\author{Emin G\"un Sirer and Robert Burgess}
\date{}

\begin{document}
\maketitle

\textit{This tutorial will show you everything you need to know about \XTeX{}. It assumes basic familiarity with \BibTeX{}. You should have \XTeX{} installed.}

\XTeX{} is a modern bibliography typesetting tool that works in conjunction with \LaTeX{}. It first builds an object hierarchy based on the bibliography database. Then it parses the text at hand to determine which objects are being cited. Then it formats these objects according to the style selected in the document, as modified by the command line options. It then produces a references section that \LaTeX{} can incorporate into the original document.

%HEVEA \begin{small}(This document is also available in \href{crosstex.pdf}{PDF}.)\end{small}



\section{Quick start}

First, make sure \XTeX{} is installed. Then, where you used to type:

\begin{small}\begin{verbatim}
$ latex paper  # Generate the .aux file
$ bibtex paper # Generate the .bbl file
$ latex paper  # Incorporate the bibliographic information
$ latex paper  # Get the labels right
\end{verbatim}\end{small}

Instead use:

\begin{small}\begin{verbatim}
$ latex paper    # Generate the .aux file
$ crosstex paper # Generate the .bbl file
$ latex paper    # Incorporate the bibliographic information
$ latex paper    # Get the labels right
\end{verbatim}\end{small}

\XTeX{} is backwards-compatible with \BibTeX{} and supports the standard \texttt{abbrv}, \texttt{alpha}, \texttt{full}, and \texttt{plain} bibliography styles.



\section{Inside the \texttt{.xtx} file}


\subsection{Objects}

Everything in \XTeX{} is an object. Every object has a \textit{key} that can be used to refer to it, and \textit{fields} containing values. Here are some objects:

\begin{small}\begin{verbatim}
@month{sep, name = "September"}

@location{rio, name = "Rio de Janeiro, Brazil"}

@author{egs, name = "Emin {G\"un} Sirer"}

@article{mypaper,
	   author = egs,
	   title = "This is My Paper",
	   journal = "Journal of Improbable Results",
	   address = rio,
	   year = 2018,
	   month = sep
}
\end{verbatim}\end{small}

The first line defines a \texttt{month} object, henceforth known as \texttt{sep}, that has a single field called \texttt{name}, which consists of the string ``September''. From here on, other objects can simply refer to \texttt{sep} wherever a month is called for, and they will be referring to this object. The second line defines a \texttt{location} object named \texttt{rio}, while the third line defines an \texttt{author} object whose name requires complicated \LaTeX{} punctuation to format properly. The final entry defines an \texttt{article}, published in Rio de Janeiro in September. Note how it refers to the previous objects by their keys. The fields of \texttt{mypaper} end up as though it had been defined thus:

\begin{small}\begin{verbatim}
@article{mypaper,
	   author = "Emin {G\"un} Sirer",
	   title = "This is My Paper",
	   journal = "Journal of Improbable Results",
	   address = "Rio de Janeiro, Brazil",
	   year = 2018,
	   month = "September"
}
\end{verbatim}\end{small}

Objects can be given multiple keys, as well. Take for example the following author:

\begin{small}\begin{verbatim}
@author{rama, name="Venugopalan Ramasubramanian"}
\end{verbatim}\end{small}

However, Rama uses Venu and Arun as names for his alter egos. So we can define his object as follows:

\begin{small}\begin{verbatim}
@author{rama = arun = venu, name="Venugopalan Ramasubramanian"}
\end{verbatim}\end{small}

Thereafter, the following are equivalent.

\begin{small}\begin{verbatim}
author = "rama and egs"
author = "arun and egs"
author = "venu and egs"
\end{verbatim}\end{small}

\subsection{Representation}

Every kind of object, such as \texttt{month}, \texttt{location}, \texttt{author}, and \texttt{article}, knows how to convert itself into a string suitable for inclusion in the references section of a scholarly publication. For example, when \texttt{mypaper} referred to \texttt{sep} in the example, the actual value assigned was ``September''. Some objects, such as \texttt{mypaper}, will produce entire bibliography entries when referred to. In fact, when generating bibliographies, \texttt{crosstex} simply prints out the string representations of all the objects cited in the document.  Each object takes note of options passed into CrossTeX and generates a string representation accordingly.

Simple, named objects, such as \texttt{month}, \texttt{location}, \texttt{author}, and others, have two forms: A long form and a short form. The object \texttt{sep} could be defined:

\begin{small}\begin{verbatim}
@month{sep,
  name = "September",
  shortname = "Sept."
}
\end{verbatim}\end{small}

By default, long names are used when generating bibliographic entries. If, however, the option \texttt{--short month} were given, the same month object would be shown as ``Sept.'' instead of ``September''. If only a name (no shortname) is specified, or, conversely, if only a shortname but no name is specified, the name given will be used in all cases. All named objects follow this pattern.

However, because person names are so complicated, \texttt{author} objects are somewhat magical. If \texttt{--short author} is specified, authors will be represented with initials and a last name---the object \texttt{egs} would be represented ``E. G. Sirer''. If a \texttt{shortname} field is explicitly given for an author, that takes precedence and can be used for special cases. 
By default, \XTeX{} is aware of many kinds of names and can correctly handle suffixes and last name modifiers such as Jr., Sr., III, IV, von, van, de, bin, and ibn. Just write the names out in full in their natural order, and \XTeX{} will render them properly, including such entries as:

\begin{small}\begin{verbatim}
@author{rvr, name = "Robbert van Renesse"}
@author{ldv, name = "Louis de Vargas, III"}
\end{verbatim}\end{small}

The \texttt{author} field is also somewhat special. Users of \BibTeX{} are familiar with specifying multiple authors as follows:

\begin{small}\begin{verbatim}
@inproceedings{credence,
  title = "{Experience with an Object
            Reputation System for 
            Peer-to-Peer Filesharing}",
  author = "Kevin Walsh and Emin {G\"un} Sirer",
  ...
}
\end{verbatim}\end{small}

Having looked up the rather non-trivial escape sequence to get the umlaut 
correct, and having made sure that all names are spelled correctly, there
is no need to repeat or cut-and-paste the same information over and over
again. The following sequence has the same effect as the previous one, 
and is much easier to maintain:


\begin{small}\begin{verbatim}
@author{kwalsh, name = "Kevin Walsh"}
@author{egs, name = "Emin {G\"un} Sirer"}
@inproceedings{credence,
  title = "{Experience with an Object
            Reputation System for 
            Peer-to-Peer Filesharing}",
  author = "kwalsh and egs",
  ...
}
\end{verbatim}\end{small}

In all other contexts, string literals and object references are quite different: \texttt{month = sep} refers to the \texttt{sep} object, which carries additional fields and is rendered differently depending on the context and options, while \texttt{month = "sep"} always generates the literal string ``sep''.

\subsection{Under the hood: References}

Referring to an object, as in \texttt{month = sep}, assigns the stringified value of the object \texttt{sep}, in this case ``September'', to the \texttt{month} field of the object. In addition, it triggers something else to happen after all the fields have been assigned: Any required or optional fields in the referring object that do not yet have values will try to inherit them from the referenced object \texttt{sep}, if it has them. The actual \texttt{month} objects defined in the included \texttt{dates.xtx}, for example, define \texttt{monthno} fields in addition to their names. This value will be pulled along into objects that refer to months, enabling such objects to be sorted by month number easily. Power-users of \BibTeX{} will note that this mechanism is essentially equivalent to the \texttt{crossref} field in \BibTeX{}. The main difference here is that \XTeX{} supports this mechanism in a uniform manner across all object fields.

This can be combined with a new, advanced feature of \XTeX{}, conditional fields, to greatly simplfy object specifications. Here is an example:

\begin{small}\begin{verbatim}
@conference{nsdi,
  shortname = "NSDI",
  longname = "Symposium on Networked System Design and Implementation", 
  [year=2006] address=SanJose, month=may,
  [year=2005] address=Boston, month=may,
  [year=2004] address=SF, month=mar,
}
\end{verbatim}\end{small}

This is a simple \texttt{conference} object whose short name is "NSDI"
and long name is "Symposium on Networked System Design and
Implementation". In addition, it carries some conditional fields; that is, 
fields that are included in the object only if they match the data in the
referring context. For instance, if the \texttt{year} field of the referring object is equal to 2006, the nsdi object additionally has the fields \texttt{address=SanJose, month=may}. In a different year, a different conditional might be triggered. Any, all, or none of the conditionals mentioned may be appropriate (although obviously in this case, the year can only have one value).

By itself, conditional fields are not very useful; their value in 
simplifying object references becomes apparent when they are used
in context. Look at the following reference:

\begin{small}\begin{verbatim}
@inproceedings{credence,
  title = "Experience with an Object Reputation System for Peer-to-Peer Filesharing",
  author = "kwalsh and egs",
  booktitle = nsdi,
  year = 2006
}
\end{verbatim}\end{small}

Conditional fields defined in the \texttt{nsdi} object will define the
address and month fields of the conference based on the reference, 
and the resulting \texttt{credence} object will know that it occurred
in San Jose in May through inheritence. This allows paper citations
to avoid common errors by allowing all conference dates and locations
to be defined in one place, and inherited correctly, without typos,
by all papers that appeared at that conference. 

If one wanted to override field inheritence for whatever reason, it would suffice to specify, say, a different month for the \texttt{credence} object. Only those fields that are missing in the referring context are inherited. Thus explicitly assigned information has precedence.

\subsection{Managing databases}

Obviously, re-inventing \texttt{sep} and \texttt{SanJose} in every
database would be exhausting. Instead, like objects can be collected
together---for example, the standard \XTeX{} distribution provides 
\texttt{dates.xtx}, which defines English month names, and 
\texttt{locations.xtx}, which defines all locations at which a 
major computer science conference was held in the recent years.
Such modules can be includes with the \texttt{@include} primitive.
For example, here is a complete \texttt{.xtx}
file based on the standard \XTeX{} distribution:

\begin{small}\begin{verbatim}
@include conferences

@author{egs, name = "Emin {G\"un} Sirer"}
@author{kwalsh, name = "Kevin Walsh"}

@inproceedings{credence,
  title = "Experience with an Object Reputation System for Peer-to-Peer Filesharing",
  author = "kwalsh and egs",
  booktitle = nsdi,
  year = 2006
}
\end{verbatim}\end{small}

This will search for \texttt{conferences.xtx} or \texttt{conferences.bib} and include it in the appropriate place before parsing the rest of the file. \XTeX{} by default looks in a standard system directory and the current directory; additional search paths can be specified with the \texttt{--dir} option. The standard \texttt{conferences.xtx} begins:

\begin{small}\begin{verbatim}
@include dates
@include locations

...
\end{verbatim}\end{small}

Thus, the \texttt{credence} object has access to well-defined locations, dates, and conference names.

It is possible for the same object to be defined multiple times under the 
same key (sometimes, this is inevitable when there are multiple bibliographic
databases involved maintained by different entities). By default, \XTeX{}
will silently ignore such definitions as long as all versions of the object
are identical. When two separate objects defined under the same key are not
identical, it points to an inconsistency in the bibliographic database, which
will cause \XTeX{} to issue a warning. Passing \XTeX{} the \texttt{--strict} flag will force it to issue such warnings even when the objects are identical,
to help facilitate people who might want to maintain databases free of
duplicate entries.

Recall that \XTeX{} enables an object to appear under multiple keys. 
This aliasing can be done for any object and can be used anywhere in
the database. The \texttt{.tex} file can cite any object by any one of
its synonymous keys. There is a strange quirk with the use of
synonymous keys stemming from a design error in LaTeX, which users
should keep in mind: If a \texttt{.tex} file cites a paper with
citation keys name1 and name2 twice, under both keys, the reference
will appear twice in the produced bibliography. As a result, feel free
to use any one of the tags to cite papers, but it's not a good idea to
use more than one tag to refer to a citation within a given paper.

When databases get exceptionally long and many elements have very similar fields---e.g., they are all in the same conference or have the same informative \texttt{category} field---you can make use of another special \XTeX{} command, \texttt{@default}. For example, here is the beginning of the \texttt{usenix.xtx} database:

\begin{small}\begin{verbatim}
@include conferences

@inproceedings{DBLP:conf/usenix/RuanP04,
  author    = {Yaoping Ruan and
               Vivek S. Pai},
  title     = {Making the "Box" Transparent: System Call Performance as
               a First-Class Result},
  booktitle = usenixg,
  year      = 2004,
  pages     = {1-14},
  ee        = {http://www.usenix.org/publications/library/proceedings/usenix04/t
ech/general/ruan.html},
  bibsource = {DBLP, http://dblp.uni-trier.de}
}

@inproceedings{DBLP:conf/usenix/CantrillSL04,
  author    = {Bryan Cantrill and
               Michael W. Shapiro and
               Adam H. Leventhal},
  title     = {Dynamic Instrumentation of Production Systems},
  booktitle = usenixg,
  year      = 2004,
  pages     = {15-28},
  ee        = {http://www.usenix.org/publications/library/proceedings/usenix04/t
ech/general/cantrill.html},
  bibsource = {DBLP, http://dblp.uni-trier.de}
}

...
\end{verbatim}\end{small}

With \texttt{@default}, it could be shortened:

\begin{small}\begin{verbatim}
@include conferences

@default booktitle = usenixg
@default year = 2004
@default bibsource = {DBLP, http://dblp.uni-trier.de}

@inproceedings{DBLP:conf/usenix/RuanP04,
  author    = {Yaoping Ruan and
               Vivek S. Pai},
  title     = {Making the "Box" Transparent: System Call Performance as
               a First-Class Result},
  pages     = {1-14},
  ee        = {http://www.usenix.org/publications/library/proceedings/usenix04/t
ech/general/ruan.html},
}

@inproceedings{DBLP:conf/usenix/CantrillSL04,
  author    = {Bryan Cantrill and
               Michael W. Shapiro and
               Adam H. Leventhal},
  title     = {Dynamic Instrumentation of Production Systems},
  pages     = {15-28},
  ee        = {http://www.usenix.org/publications/library/proceedings/usenix04/t
ech/general/cantrill.html},
}

...
\end{verbatim}\end{small}

Later in the file are entries with different years. A new \texttt{@default} command takes precedence over the first:

\begin{small}\begin{verbatim}
...

@default year = 2003

@inproceedings{DBLP:conf/usenix/PadioleauR03,
  author    = {Yoann Padioleau and
               Olivier Ridoux},
  title     = {A Logic File System},
  pages     = {99-112},
  ee        = {http://www.usenix.org/events/usenix03/tech/padioleau.html},
}

@inproceedings{DBLP:conf/usenix/DouglisI03,
  author    = {Fred Douglis and
               Arun Iyengar},
  title     = {Application-specific Delta-encoding via Resemblance Detection},
  pages     = {113-126},
  ee        = {http://www.usenix.org/events/usenix03/tech/douglis.html},
}

...
\end{verbatim}\end{small}

As with field values inherited from references objects, field values
inherited from \texttt{default} definitions have lower precedence. Any
object that explicitly assigns a value to a field will override any 
\texttt{default} definitions in effect at that point in the bibliography.


\section{Invoking \texttt{crosstex}}

The \texttt{crosstex} program will search for each file listed on the command line, including versions with \texttt{.aux}, \texttt{.xtx}, and \texttt{.bib} appended. It will then convert each file to a \texttt{.bbl} file containing the appropriate bibliography. Generally, no options are necessary except when tweaking a bibliography's style. There are many more options than are shown here; run \texttt{crosstex --help} for a complete list with explanations.

\begin{description}
\item[\texttt{-d DIR}, \texttt{--dir=DIR}]
By default, \XTeX{} searches for files in the current directory and a central system directory such as \texttt{/usr/share/texmf/crosstex}. This option adds a new directory that will be searched before them.
\end{description}


\subsection{Debugging options}

\begin{description}
\item[\texttt{--quiet}]
Prevents errors from being printed.

\item[\texttt{--strict}]
Causes warnings to be printed.
\end{description}


\subsection{Style options}

\begin{description}
\item[\texttt{--cite=CITE}]
Generate a citation, exactly as though \texttt{\textbackslash{}cite\{CITE\}} had appeared, including the \LaTeX{} convention \texttt{*} for ``cite everything''.

\item[\texttt{--style=STYLE}]
Specify the bibliography style, as though \texttt{\textbackslash{}bibliographystyle\{STYLE\}} had appeared.

\item[\texttt{-s FIELD}, \texttt{--sort=FIELD}]
Sort entries by the specified field. Multiple sort orders are applied (stably) in the order specified, e.g. \texttt{-s year -s author} will cause elements to be grouped primarily by author and sub-grouped by year.

\item[\texttt{-S FIELD}, \texttt{--reverse-sort=FIELD}]
Exactly as --sort, but sort by descending field values rather than ascending.

\item[\texttt{--no-sort}]
Leave the bibliography in citation order.

\item[\texttt{--heading=FIELD}]
Use the specified field to divide the bibliography into several sections. For example, with \texttt{--heading category}, all entries with the same category will be collected together under a heading containing the their mutual \texttt{category} field. Another useful one could be \texttt{--heading year}. This has no influence on sort order, but simply partitions the sorted elements.

\item[\texttt{--short=TYPE}]
Where TYPE is some string-like object (\texttt{author}, \texttt{conference}, \texttt{conferencetrack}, \texttt{country}, \texttt{journal}, \texttt{month}, \texttt{state}, \texttt{string}, or \texttt{workshop}), abbreviate all objects of that type.

\item[\texttt{--add-in}]
Add ``In'' for \texttt{article} entries.

\item[\texttt{--add-proceedings}]
Add ``In Proceedings of'' for conference and workshop publications.

\item[\texttt{--add-proc}]
Add ``Proc. of'' for conference and workshop publications.

\item[\texttt{-l FIELD}, \texttt{--link=FIELD}, \texttt{--no-link}]
For each field given as \texttt{-l \textrm{\textit{fieldname}}}, if an entry has that field and it appears to be a URL, it will be included in the bibliography as a link named after the field. For example, with \texttt{-l Abstract}, entries with an \texttt{abstract} field consisting of a URL will have a link to that URL, which appears as \textsc{Abstract}. To empty this list of fields (i.e. cause no links to be generated), use \texttt{--no-link}.

Normally, no links are generated; when generating HTML, the default fields are Abstract, URL, PS, PDF, HTML, DVI, TEX, BIB, FTP, HTTP, and RTF. If this option is used, the document use \texttt{\textbackslash{}usepackage\{hyperref\}}.

\item[\texttt{--abstract}, \texttt{--no-abstract}]
Include (or do not include) paper abstracts in the bibliography when they are provided. If \texttt{--link abstract} has also been specified, and the abstract was decided to consist of a URL, it will appear as a link and not be included also by this option. If, however, it existed but did not consist of a link, it will appear in full as normal.

\item[\texttt{--keywords}, \texttt{--no-keywords}]
Include (or do not include) keywords in the bibliography when they are provided.

\end{description}


\subsection{Converting back to \BibTeX{}}

When invoked as \texttt{xtx2bib} or given the \texttt{--xtx2bib} option, the database information will be collected and re-output as a \texttt{.bib} file for processing by \BibTeX{}.


\subsection{Generating HTML}

When invoked as \texttt{xtx2html} or given the \texttt{--xtx2html} option, a standalone bibliography will be generated and translated to HTML. This can be used to conveniently prepare an entire bibliographic database (say, of your own publications) or the citations of a paper for inclusion in a web site such as a home page. By defaults \texttt{xtx2html} turns on some pretty non-traditional options, which can of course be changed by command-line options (\texttt{xtx2html --style plain \textrm{\textit{files}}}, for example, makes something very tame).



\section{Inside the \texttt{.tex} file}

The \texttt{.tex} source file never needs to know you're using \XTeX{}, because it is completely backwards compatible. The \texttt{\textbackslash{}bibliographystyle\{\textrm{\textit{foo}}\}} command will cause the equivalent of the command-line option \texttt{--style \textrm{\textit{foo}}}.

However, the \texttt{\textbackslash{}bibliographystyle} command can cause other magic as well. Arbitrary command line options may be specified after the style name, separated by commas. For example:

\begin{small}\begin{verbatim}
\bibliographystyle{plain,--add-in,--add-proceedings,--short,author}
\end{verbatim}\end{small}

This allows documents to have fine-grained control over styling. Run-time options will still take precedence over document defaults.



\section{Extending \XTeX{}}
\label{extending}

\XTeX{} is designed to be easy to extend with very trivial knowledge of Python. Before continuing, it is very important to have a look at the standard object types and fields in Section~\ref{objects}, which are already supported by \XTeX{}. New objects or fields are defined by editing the file \texttt{crosstexobjects.py}, which is typically installed in \texttt{/usr/share/texmf/crosstex}.

To create a new field for a particular object type, find its definition (e.g., the section defining the \texttt{string} object begins \texttt{class string}). Most objects already define some fields; simply copy that syntax for your own field. If a field is assigned \texttt{None} to begin with, it will be required; by assigning it some default value, even the empty string, it becomes optional.

To make an optional field required or a required field optional, simply assign it \texttt{None} or some default, respectively, in the class where you want the change.

To create an entirely new class \textit{foo} which is identical to a current one named \textit{bar}, add the following to the end of \texttt{crosstexobjects.py}.

\begin{small}\begin{verbatim}
class foo(misc):
    pass
\end{verbatim}\end{small}

If \textit{foo} should have different required or optional fields form \textit{bar}, define them as noted above. (The \texttt{pass} is only necessary if the body of the \texttt{class} definition is empty, e.g. it does not set the values of any fields.)

If you are more advanced at Python programming and want to change the behavior of objects, simply look at the ones that exist. The string value of an object is obtained, appropriately enough, from its \texttt{\_\_str\_\_} method, should you want to change the appearance of any objects. Similarly, every object has access to an \texttt{\_options} member containing the options set on the command line.

Happy hacking!



\section{Standard object types}
\label{objects}

These are the kinds of objects \XTeX{} knows about by default. For information about extending this notion, see Extending \XTeX{} in Section~\ref{extending}.

\begin{description}

\item[\texttt{string}]
\textsc{Required:} \texttt{name} and/or \texttt{shortname} (\texttt{longname} is an alias for \texttt{name}.)
\textsc{Relevant arguments:} \texttt{--short}

\item[\texttt{author}] As \texttt{string}, except:
\textsc{Optional:} \texttt{address}, \texttt{affiliation}, \texttt{email}, \texttt{institution}, \texttt{organization}, \texttt{phone}, \texttt{school}, \texttt{url}

\item[\texttt{state}] As \texttt{string}, except:
\textsc{Optional:} \texttt{country}

\item[\texttt{country}] As \texttt{string}.

\item[\texttt{location}]
\textsc{Optional:} \texttt{city}, \texttt{state}, \texttt{country}

\item[\texttt{month}] As \texttt{string}.

\item[\texttt{journal}] As \texttt{string}.

\item[\texttt{misc}]
\textsc{Optional:}
\texttt{abstract},
\texttt{address},
\texttt{affiliation},
\texttt{annote},
\texttt{author},
\texttt{bib},
\texttt{bibsource},
\texttt{booktitle},
\texttt{category},
\texttt{chapter},
\texttt{contents},
\texttt{copyright},
\texttt{crossref},
\texttt{doi},
\texttt{dvi},
\texttt{edition},
\texttt{editor},
\texttt{ee},
\texttt{ftp},
\texttt{howpublished},
\texttt{html},
\texttt{http},
\texttt{institution},
\texttt{isbn},
\texttt{issn},
\texttt{journal},
\texttt{key},
\texttt{keywords},
\texttt{language},
\texttt{lccn},
\texttt{location},
\texttt{month},
\texttt{monthno},
\texttt{mrnumber},
\texttt{note},
\texttt{number},
\texttt{organization},
\texttt{pages},
\texttt{pdf},
\texttt{price},
\texttt{ps},
\texttt{publisher},
\texttt{rtf},
\texttt{school},
\texttt{series},
\texttt{size},
\texttt{title},
\texttt{type},
\texttt{url},
\texttt{volume},
\texttt{year}
\textsc{Relevant arguments:} \texttt{--cite-by}, \texttt{--titlecase}, \texttt{--link}, \texttt{--abstract}, \texttt{--keywords}

\item[\texttt{article}] As \texttt{misc}, except:
\textsc{Required:} \texttt{author}, \texttt{title}, \texttt{journal}, \texttt{year}
\textsc{Relevant arguments:} \texttt{--add-in}

\item[\texttt{book}] As \texttt{misc}, except:
\textsc{Required:} \texttt{author} and/or \texttt{editor}, \texttt{title}, \texttt{publisher}, \texttt{year}

\item[\texttt{booklet}] As \texttt{misc}, except:
\textsc{Required:} \texttt{title}

\item[\texttt{inbook}] As \texttt{misc}, except:
\textsc{Required:} \texttt{author} and/or \texttt{editor}, \texttt{title}, \texttt{chapter} and/or \texttt{pages}, \texttt{publisher}, \texttt{year}

\item[\texttt{incollection}] As \texttt{misc}, except:
\textsc{Required:} \texttt{author}, \texttt{title}, \texttt{booktitle}, \texttt{publisher}, \texttt{year}

\item[\texttt{inproceedings}] As \texttt{misc}, except:
\textsc{Required:} \texttt{author}, \texttt{title}, \texttt{booktitle}, \texttt{year}
\textsc{Relevant arguments:} \texttt{--add-proceedings}, \texttt{--add-proc}

\item[\texttt{manual}] As \texttt{misc}, except:
\textsc{Required:} \texttt{title}

\item[\texttt{thesis}] As \texttt{misc}, except:
\textsc{Required:} \texttt{author}, \texttt{title}, \texttt{school}, \texttt{year}

\item[\texttt{mastersthesis}] As \texttt{thesis}.

\item[\texttt{phdthesis}] As \texttt{thesis}.

\item[\texttt{patent}] As \texttt{misc}.

\item[\texttt{proceedings}] As \texttt{misc}, except:
\textsc{Required:} \texttt{title}, \texttt{year}

\item[\texttt{collection}] As \texttt{proceedings}.

\item[\texttt{techreport}] As \texttt{misc}, except:
\textsc{Required:} \texttt{author}, \texttt{title}, \texttt{institution}, \texttt{year}

\item[\texttt{unpublished}] As \texttt{misc}, except:
\textsc{Required:} \texttt{author}, \texttt{title}, \texttt{note}

\item[\texttt{conference}] As \texttt{string}, except:
\textsc{Optional:}
\texttt{address},
\texttt{crossref},
\texttt{editor},
\texttt{institution},
\texttt{isbn},
\texttt{key},
\texttt{keywords},
\texttt{language},
\texttt{location},
\texttt{month},
\texttt{publisher},
\texttt{url},
\texttt{year}

\item[\texttt{conferencetrack}] As \texttt{conference}, except:
\textsc{Optional:} \texttt{conference}

\item[\texttt{workshop}] As \texttt{conferencetrack}.

\item[\texttt{rfc}] As \texttt{misc}, except:
\textsc{Required:} \texttt{author}, \texttt{title}, \texttt{number}, \texttt{month}, \texttt{year}

\end{description}

\end{document}
