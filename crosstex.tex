\documentclass{article}
\usepackage{hyperref}

%HEVEA \renewcommand{\TeX}{TeX}
%HEVEA \renewcommand{\LaTeX}{LaTeX}
\newcommand{\XTX}{Cross\TeX}
\newcommand{\BibTeX}{\textsc{Bib}\TeX}

%HEVEA \renewcommand{\par}{\begin{rawhtml}<p>\end{rawhtml}}

\title{\XTX{} Tutorial}
\author{Emin G\"un Sirer and Robert Burgess}
\date{}

\begin{document}
\maketitle

\textit{
This tutorial will show you everything you need to know about \XTX{}. It
assumes basic familiarity with \BibTeX{}. You should have \XTX{}
installed.
%HEVEA \begin{small}(This document is also available in \href{crosstex.pdf}{PDF}.)\end{small}
}

\XTX{} is a modern bibliography typesetting tool that works in
conjunction with \LaTeX{}. It first builds an object hierarchy based on
the bibliography database. Then it parses the text at hand to determine
which objects are being cited. Then it formats these objects according
to the style selected in the document, as modified by the command
line options. It then produces a references section that \LaTeX{} can
incorporate into the original document.



\section{Quick start}

First, make sure \XTX{} is installed. Then, where you used to type:

\begin{small}\begin{verbatim}
$ latex paper  # Generate the .aux file
$ bibtex paper # Generate the .bbl file
$ latex paper  # Incorporate the bibliographic information
$ latex paper  # Get the labels right
\end{verbatim}\end{small}

Instead use:

\begin{small}\begin{verbatim}
$ latex paper    # Generate the .aux file
$ crosstex paper # Generate the .bbl file
$ latex paper    # Incorporate the bibliographic information
$ latex paper    # Get the labels right
\end{verbatim}\end{small}

\XTX{} is backwards-compatible with \BibTeX{} and supports the standard
\texttt{abbrv}, \texttt{alpha}, \texttt{full}, and \texttt{plain}
bibliography styles.

\section{Defining Objects: Inside the \texttt{.xtx} file\label{sec:objects}}

\subsection{Objects}

Everything in \XTX{} is an object. Every object has a \textit{key} that
can be used to refer to it, and \textit{fields} containing values. Here
are some objects:

\begin{small}\begin{verbatim}
@month{sep, name = "September"}

@location{rio, name = "Rio de Janeiro, Brazil"}

@author{egs, name = "Emin {G\"un} Sirer"}

@article{mypaper,
	   author = egs,
	   title = "This is My Paper",
	   journal = "Journal of Improbable Results",
	   address = rio,
	   year = 2018,
	   month = sep
}
\end{verbatim}\end{small}

The first line defines a \texttt{month} object, henceforth known as
\texttt{sep}, that has a single field called \texttt{name}, which consists
of the string ``September''. From here on, other objects can simply refer
to \texttt{sep} wherever a month is called for, and they will be referring
to this object. The second line defines a \texttt{location} object named
\texttt{rio}, while the third line defines an \texttt{author} object whose
name requires complicated \LaTeX{} punctuation to format properly. The
final entry defines an \texttt{article}, published in Rio de Janeiro in
September. Note how it refers to the previous objects by their keys. The
fields of \texttt{mypaper} end up as though it had been defined thus:

\begin{small}\begin{verbatim}
@article{mypaper,
	   author = "Emin {G\"un} Sirer",
	   title = "This is My Paper",
	   journal = "Journal of Improbable Results",
	   address = "Rio de Janeiro, Brazil",
	   year = 2018,
	   month = "September"
}
\end{verbatim}\end{small}

Objects can be given multiple keys, as well. Take for example the
following author:

\begin{small}\begin{verbatim}
@author{rama, name="Venugopalan Ramasubramanian"}
\end{verbatim}\end{small}

However, Rama uses Venu and Arun as names for his alter egos. So we can
define his object as follows:

\begin{small}\begin{verbatim}
@author{rama = arun = venu, name="Venugopalan Ramasubramanian"}
\end{verbatim}\end{small}

Thereafter, the following are equivalent.

\begin{small}\begin{verbatim}
author = "rama and egs"
author = "arun and egs"
author = "venu and egs"
\end{verbatim}\end{small}

\subsection{Representation}

Every kind of object, such as \texttt{month}, \texttt{location},
\texttt{author}, and \texttt{article}, knows how to convert itself into
a string suitable for inclusion in the references section of a scholarly
publication. For example, when \texttt{mypaper} referred to \texttt{sep}
in the example, the actual value assigned was ``September''. Some objects,
such as \texttt{mypaper} itself, will produce entire bibliography
entries when referred to. In fact, when generating bibliographies,
\texttt{crosstex} simply prints out the string representations of all
the objects cited in the document.  Each object takes note of options
passed into \XTX{} and generates a string representation accordingly.

Simple, named objects, such as \texttt{month}, \texttt{location},
\texttt{author}, and others, have two forms: A long form and a short
form. The object \texttt{sep} could be defined:

\begin{small}\begin{verbatim}
@month{sep,
  name = "September",
  shortname = "Sept."
}
\end{verbatim}\end{small}

By default, long names are used when generating bibliographic entries. If,
however, the option \texttt{--short month} were given, the same month
object would be shown as ``Sept.'' instead of ``September''. If only a
name (no shortname) is specified, or, conversely, if only a shortname
but no name is specified, the name given will be used in all cases. All
named objects follow this pattern.

However, because person names are so complicated, \texttt{author}
objects are somewhat magical. If \texttt{--short author} is specified,
authors will be represented with initials and a last name---the object
\texttt{egs} would be represented ``E. G. Sirer''. If a \texttt{shortname}
field is explicitly given for an author, that takes precedence and can
be used for special cases.  By default, \XTX{} is aware of many kinds of
names and can correctly handle suffixes and last name modifiers such as
Jr., Sr., III, IV, von, van, de, bin, and ibn. Just write the names out
in full in their natural order, and \XTX{} will render them properly,
including such entries as:

\begin{small}\begin{verbatim}
@author{rvr, name = "Robbert van Renesse"}
@author{ldv, name = "Louis de Vargas, III"}
\end{verbatim}\end{small}

The \texttt{author} field is also somewhat special. Users of \BibTeX{}
are familiar with specifying multiple authors as follows:

\begin{small}\begin{verbatim}
@inproceedings{credence,
  title = "{Experience with an Object
            Reputation System for 
            Peer-to-Peer Filesharing}",
  author = "Kevin Walsh and Emin {G\"un} Sirer",
  ...
}
\end{verbatim}\end{small}

Having looked up the rather non-trivial escape sequence to get the umlaut
correct, and having made sure that all names are spelled correctly, there
is no need to repeat or cut-and-paste the same information over and over
again. The following sequence has the same effect as the previous one,
and is much easier to maintain:

\begin{small}\begin{verbatim}
@author{kwalsh, name = "Kevin Walsh"}
@author{egs, name = "Emin {G\"un} Sirer"}
@inproceedings{credence,
  title = "{Experience with an Object
            Reputation System for 
            Peer-to-Peer Filesharing}",
  author = "kwalsh and egs",
  ...
}
\end{verbatim}\end{small}

In all other contexts, string literals and object references are quite
different: \texttt{month = sep} refers to the \texttt{sep} object, which
carries additional fields and is rendered differently depending on the
context and options, while \texttt{month = "sep"} always generates the
literal string ``sep''.

\subsection{Under the hood: References}

Referring to an object, as in \texttt{month = sep}, assigns the
stringified value of the object \texttt{sep}, in this case ``September'',
to the \texttt{month} field of the object. In addition, it triggers
something else to happen after all the fields have been assigned:
Any required or optional fields in the referring object that do not
yet have values will try to inherit them from the referenced object
\texttt{sep}, if it has them. The actual \texttt{month} objects defined
in the included \texttt{dates.xtx}, for example, define \texttt{monthno}
fields in addition to their names. This value will be pulled along into
objects that refer to months, enabling such objects to be sorted by month
number easily. Power-users of \BibTeX{} will note that this mechanism is
essentially equivalent to the \texttt{crossref} field in \BibTeX{}. The
main difference here is that \XTX{} supports this mechanism in a uniform
manner across all object fields.

This can be combined with a new, advanced feature of \XTX{}, conditional
fields, to greatly simplify object specifications. Here is an example:

\begin{small}\begin{verbatim}
@conference{nsdi,
  shortname = "NSDI",
  longname = "Symposium on Networked System Design and Implementation", 
  [year=2006] address=SanJose, month=may,
  [year=2005] address=Boston, month=may,
  [year=2004] address=SF, month=mar,
}
\end{verbatim}\end{small}

This is a simple \texttt{conference} object whose short name is
``NSDI'' and long name is ``Symposium on Networked System Design and
Implementation''. In addition, it carries some conditional fields; that
is, fields that are included in the object only if they match the data
in the referring context. For instance, if the \texttt{year} field of the
referring object is equal to 2006, the \texttt{nsdi} object additionally
has the fields \texttt{address=SanJose, month=may}. In a different year,
a different conditional might be triggered. Any, all, or none of the
conditionals mentioned may be appropriate (although obviously in this
case, the year can only have one value).

By itself, conditional fields are not very useful; their value in
simplifying object references becomes apparent when they are used in
context. Look at the following reference:

\begin{small}\begin{verbatim}
@inproceedings{credence,
  title = "Experience with an Object Reputation System for Peer-to-Peer Filesharing",
  author = "kwalsh and egs",
  booktitle = nsdi,
  year = 2006
}
\end{verbatim}\end{small}

Conditional fields defined in the \texttt{nsdi} object will define
the address and month fields of the conference based on the reference,
and the resulting \texttt{credence} object will know that it occurred
in San Jose in May through inheritance. This allows paper citations to
avoid common errors by allowing all conference dates and locations to
be defined in one place, and inherited correctly, without typos, by all
papers that appeared at that conference.

If one wanted to override field inheritance for whatever reason, it would
suffice to specify, say, a different month for the \texttt{credence}
object. Only those fields that are missing in the referring context are
inherited. Thus explicitly assigned information has precedence.

\subsection{Including other databases}

Obviously, re-inventing \texttt{sep} and \texttt{SanJose} in every
database would be exhausting. Instead, similar objects can be
collected together---for example, the standard \XTX{} distribution
provides \texttt{dates.xtx}, which defines English month names, and
\texttt{locations.xtx}, which defines all locations at which a major
computer science conference was held in the recent years.  Such modules
can be includes with the \texttt{@include} primitive.  For example,
here is a complete \texttt{.xtx} file based on the standard \XTX{}
distribution:

\begin{small}\begin{verbatim}
@include conferences-cs

@author{egs, name = "Emin {G\"un} Sirer"}
@author{kwalsh, name = "Kevin Walsh"}

@inproceedings{credence,
  title = "Experience with an Object Reputation System for Peer-to-Peer Filesharing",
  author = "kwalsh and egs",
  booktitle = nsdi,
  year = 2006
}
\end{verbatim}\end{small}

This will search for \texttt{conferences-cs.xtx} or \texttt{conferences-cs.bib}
and include it in the appropriate place before parsing the rest of the
file. \XTX{} by default looks in a standard system directory and the
directory containing the database or document being processed; additional
search paths can be specified with the \texttt{--dir} option. The standard
\texttt{conferences-cs.xtx} begins:

\begin{small}\begin{verbatim}
@include dates
@include locations

...
\end{verbatim}\end{small}

Thus, the \texttt{credence} object has access to well-defined locations,
dates, and conference names.

On startup, \texttt{crosstex} will read in the database \texttt{standard},
which in the distribution pulls in the \texttt{dates} database and
some information to help accurate formatting of titles.  Having the
dates available by default is necessary for backwards compatibility
with \BibTeX{}; the administrator may also edit the standard database
in order to automatically include additional important files such as an
institution-local bibliography.

It is possible for the same object to be defined multiple times under
the same key (sometimes, this is inevitable when there are multiple
bibliographic databases involved maintained by different entities). By
default, \XTX{} will silently ignore such definitions as long as all
versions of the object are identical. When two separate objects defined
under the same key are not identical, it points to an inconsistency
in the bibliographic database, which will cause \XTX{} to issue a
warning. Passing \XTX{} the \texttt{--strict} flag will force it to issue
such warnings even when the objects are identical, to help facilitate
people who might want to maintain databases free of duplicate entries.

\subsection{Extending objects}

Occasionally it is useful to add information to an object that already
exists.  For example, say you have a paper to cite that appeared in
USENIX 2006, but the system database only has the following information
about the USENIX conference:

\begin{small}\begin{verbatim}
@conference{usenix = usenixg,
  shortname = "USENIX",
  longname = "USENIX Annual Technical Conference",
  [year=2005] address=Anaheim, month=apr,
  [year=2004] address=Boston, month=jun,
  [year=2003] address=SanAntonio, month=jun,
  [year=2002] address=Monterey, month=jun,
  [year=2001] address=Boston, month=jun,
  [year=2000] address=SanDiego, month=jun,
  [year=1999] address=Monterey, month=jun,
  [year=1996] address=SanDiego, month=jan,
}
\end{verbatim}\end{small}

Obviously, the best solution is to add the following line to the entry
in the system conferences database:

\begin{small}\begin{verbatim}
  [year=2006] address=Boston, month=may,
\end{verbatim}\end{small}

However, you may not have permission to edit the database.  Now there are
two options: Cut-and-paste the usenix object into some local database with
a new name so there is no conflict, or put the address and month directly
into the paper's entry.  Neither one is a good solution.  What you want is
to be able to extend the \texttt{usenix} object even though it is in
another database you can't edit.

Enter the \texttt{@extend} primitive.  The following solves the example:

\begin{small}\begin{verbatim}
@extend{usenix,
  [year=2006] address=Boston, month=may,
}
\end{verbatim}\end{small}

An \texttt{@extend} entry looks just like an object definition.  However,
rather than defining a new object, the object with the specified key is
re-built with the information provided, inheriting its old fields with
lower priority so that extended fields take precedence.

It is possible to create new aliases along the way.  Simply list aliases
in the exact same syntax as for object definition; all the aliases listed
must either be a new, unused alias or refer to the same unique object.
After the object is extended, all the aliases mentioned will be handles
to refer to the newly-extended object.  This can be useful for defining
shorter, easier-to-remember names for database objects.

\subsection{Default fields}

When databases get exceptionally long and many elements have very
similar fields---e.g., they are all in the same conference or have the
same informative \texttt{category} field---you can make use of another
special \XTX{} command, \texttt{@default}. For example, here is the
beginning of the \texttt{usenix.xtx} database:

\begin{small}\begin{verbatim}
@include conferences-cs

@inproceedings{DBLP:conf/usenix/RuanP04,
  author    = {Yaoping Ruan and
               Vivek S. Pai},
  title     = {Making the "Box" Transparent: System Call Performance as
               a First-Class Result},
  booktitle = usenixg,
  year      = 2004,
  pages     = {1-14},
  ee        = {http://www.usenix.org/publications/library/proceedings/usenix04/t
ech/general/ruan.html},
  bibsource = {DBLP, http://dblp.uni-trier.de}
}

@inproceedings{DBLP:conf/usenix/CantrillSL04,
  author    = {Bryan Cantrill and
               Michael W. Shapiro and
               Adam H. Leventhal},
  title     = {Dynamic Instrumentation of Production Systems},
  booktitle = usenixg,
  year      = 2004,
  pages     = {15-28},
  ee        = {http://www.usenix.org/publications/library/proceedings/usenix04/t
ech/general/cantrill.html},
  bibsource = {DBLP, http://dblp.uni-trier.de}
}

...
\end{verbatim}\end{small}

With \texttt{@default}, it could be shortened:

\begin{small}\begin{verbatim}
@include conferences-cs

@default booktitle = usenixg
@default year = 2004
@default bibsource = {DBLP, http://dblp.uni-trier.de}

@inproceedings{DBLP:conf/usenix/RuanP04,
  author    = {Yaoping Ruan and
               Vivek S. Pai},
  title     = {Making the "Box" Transparent: System Call Performance as
               a First-Class Result},
  pages     = {1-14},
  ee        = {http://www.usenix.org/publications/library/proceedings/usenix04/t
ech/general/ruan.html},
}

@inproceedings{DBLP:conf/usenix/CantrillSL04,
  author    = {Bryan Cantrill and
               Michael W. Shapiro and
               Adam H. Leventhal},
  title     = {Dynamic Instrumentation of Production Systems},
  pages     = {15-28},
  ee        = {http://www.usenix.org/publications/library/proceedings/usenix04/t
ech/general/cantrill.html},
}

...
\end{verbatim}\end{small}

Later in the file are entries with different years. A new
\texttt{@default} command takes precedence over the first:

\begin{small}\begin{verbatim}
...

@default year = 2003

@inproceedings{DBLP:conf/usenix/PadioleauR03,
  author    = {Yoann Padioleau and
               Olivier Ridoux},
  title     = {A Logic File System},
  pages     = {99-112},
  ee        = {http://www.usenix.org/events/usenix03/tech/padioleau.html},
}

@inproceedings{DBLP:conf/usenix/DouglisI03,
  author    = {Fred Douglis and
               Arun Iyengar},
  title     = {Application-specific Delta-encoding via Resemblance Detection},
  pages     = {113-126},
  ee        = {http://www.usenix.org/events/usenix03/tech/douglis.html},
}

...
\end{verbatim}\end{small}

As with field values inherited from references objects, field values
inherited from \texttt{default} definitions have lower precedence. Any
object that explicitly assigns a value to a field will override any
\texttt{default} definitions in effect at that point in the bibliography.

\subsection{Comments}

Comments in \XTX{} can be accomplished in a number of ways.  Simple comments
that last until the end of the line are introduced with a \texttt{\%} character.
For example:

\begin{small}\begin{verbatim}
@include conferences-cs   % Because we need nsdi later on
\end{verbatim}\end{small}

More involved, potentially multi-line comments appear as their own kind of
primitive:

\begin{small}\begin{verbatim}
@comment "This is a database for...
Yadda yadda...
Now I've said enough."
\end{verbatim}\end{small}

This syntax can also take advantage of the
\verb"{"\textit{\ldots}\verb"}" form of strings in order to comment
out whole objects or sets of objects, since braces are counted and matched
correctly so that embedded strings don't accidentally end the comment.

\begin{small}\begin{verbatim}
@comment {
  @inproceedings{bad,
    title = "Some paper we want to temporarily comment out",
    author = "Somebody and Somebody Else",
    ...
  }
}
\end{verbatim}\end{small}

\section{Citing References\label{sec:citing}}

Once you have defined your objects in the \texttt{.xtx} file, you may
refer to them in your \texttt{.tex} file. Such references are known as
citations, and are accomplished with the \verb"\cite" command in \LaTeX{}.
\XTX{} supports two kinds of citations, both backwards compatible with
standard \LaTeX{} citations.

\subsection{Plain Citations}

The first type are plain citations based on an object key. Plain citations
simply take a comma-separated list of object keys, and cite the objects
whose keys, specified in the XTX file as the first item following
the object definition, match the cited key. For instance, given the
definitions above, the following are examples of plain citations:

\begin{small}\begin{verbatim}
Credence~\cite{credence} provides a reputation
system for peer-to-peer systems.  These two
papers~\cite{DBLP:conf/usenix/PadioleauR03,DBLP:conf/usenix/DouglisI03}
appeared at the Usenix annual conference.
\end{verbatim}\end{small}

The key used in a plain citation must match, exactly, the key used in
the object definition. The matching is case sensitive, so ``foo'' and
``FOO'' refer to different objects.

Recall that \XTX{} enables an object to appear under multiple keys.
This aliasing can be done for any object and can be used anywhere in the
database. The \texttt{.tex} file can cite any object by any one of its
synonymous keys. There is a strange quirk with the use of synonymous
keys stemming from a design error in \LaTeX{}, which users should
keep in mind: \LaTeX{} assumes that each object has only one key, and
thus citing the same object under two different keys would require it
appearing twice in the references.  Therefore, authors must be careful
to cite each paper by only one of its aliases.  Fortunately, it does not
matter which alias is used in the document, so long as it is consistent,
and it is easy for \XTX{} to detect when multiple aliases are being used,
so an error message will appear.

Overall, there really are not that many frills to plain citations. They
work exactly the way one would imagine they would. Their big drawback,
however, is that you need to remember the precise key for every
object you want to cite. Often, this requires browsing database files,
searching for author names and keywords in the title so you can figure
out whether you named the key ``credence'' or ``credence\_nsdi04'' or
``nsdi04\_credence''. Even though the standard libraries that come with
\XTX{} follow the uniform naming rule from the DBLP database, figuring
out the uniform name still requires knowing the authors and the year,
which often requires a Google search.  To make the citation process
even easier and simpler, \XTX{} supports a second kind of citation,
where the user need not recall the object key precisely.

\subsection{Constrained Citations}

The second kind of citation that \XTX{} supports is known as a
\textit{constrained citation}.  Constrained citations enable the user
to cite a paper by specifying pieces of information about the reference
that uniquely identify it. For instance, suppose you want to reference
that paper I wrote in 1999 on how to split up virtual machines, and
you remember that it appeared at SOSP. You could search your database
for some partial terms that appear in the entry (e.g. 1999, sosp,
sirer), copy the key for the entry, and issue a plain citation using
that precise key. This is what many \BibTeX{} users do without thinking.
But it is a lot of pointless boring work, and computers were supposed
to automate boring tasks. That's where constrained citations come in.

A constrained citation begins with an exclamation point, and specifies
a series of colon-separated terms that identify the reference being
cited. Some examples of constrained citations are:

\begin{small}\begin{verbatim}
\cite{!author=sirer:title=virtual:year=1999}
\cite{!author=sirer:title=virtual:title=machines:year=1999}
\cite{!author=sirer:author=walsh:year=2006}
\end{verbatim}\end{small}

Colons separate constraints. Each constraint identifies a field that the
reference must have, as well as a string that should appear somewhere
within that named field.

Each string in a constrained citation is checked for a partial match in
the corresponding field. So ``author=smith'' will match both ``Smith''
and ``Smithson.''

Sometimes, there are multiple constraints that apply to the same
field. Specifying the same field multiple times, as in the second and
third examples above, is perfectly acceptable, but gets tedious. So \XTX{}
provides a way to specify multiple constraints for the same field; every
word separated by a ``-'' sign is treated as a separate constraint. So
the examples above can be shortened down to:

\begin{small}\begin{verbatim}
\cite{!author=sirer:title=virtual:year=1999}
\cite{!author=sirer:title=virtual-machines:year=1999}
\cite{!author=sirer-walsh:year=2006}
\end{verbatim}\end{small}

Multiple constraints within a given field are not ordered and can appear
anywhere in the string, so ``virtual-machines'' will match ``virtual
machines,'' as well as ``machines virtual,'' and even ``building a
machineshop virtually.''

Several shorthands make constrained citations even easier to specify by
providing defaults for fieldnames. If the fieldnames are missing, the
first constraint defaults to ``author.'' The second constraint defaults to
``title'' if the value is not numeric; if it is, it defaults to ``year.''
Finally, the last constraint defaults to ``year.'' So the examples above
can be shortened even further:

\begin{small}\begin{verbatim}
\cite{!sirer:virtual:1999}
\cite{!sirer:virtual-machines:1999}
\cite{!sirer-walsh:2006}
\end{verbatim}\end{small}

Two caveats are worth remembering about constrained citations. First,
the citation needs to be uniquely identifiable. If the constraints you
specified match more than one object, \XTX{} will print an error and
identify the matching objects. You can then specify more constraints until
you have nailed down the reference you had in mind or switch to a plain
citation. Second, due to a limitation in \LaTeX{} mentioned above for plain
citations, referring to the same paper through different constraints
(e.g.  ``!sirer:virtual:1999'' and ``!sirer:virtual-machines:1999'')
will cause an error so the paper does not appear twice in the references
section. For each paper, you should figure out the constraints you had
in mind and stick to them throughout your document.

Overall, constrained citations are a very convenient way to cite papers
without having to look anything up. They fit naturally to the way people
recall citations. The concept was entirely lifted from Norman Ramsey's
\texttt{nbibtex} system.

\subsection{Citation Appearance}

How the citation itself appears is controlled by the citation style,
and is controlled by options specified to crosstex either in the \LaTeX{}
file or passed on the command line during invocation. The argument to the
\texttt{--cite-by} option determines how the citations appear in the body
of the text.  There are three possible arguments to \texttt{--cite-by}.

\texttt{numeric} produces citations that appear like this ``[1]''. The
numbers correspond to the location of the entry in the references
section. Another option determines how the references section is sorted
(e.g. in the order cited, alphabetized by author, or sorted according
to any field of choice), and thus affects the particular number used to
refer to a particular reference.

\texttt{initials} produces citations that appear like this ``[WS04]''. The
particular rule used to derive the initials from author names is
somewhat complex, but roughly speaking, the citation string consists of
the first initials of the authors last names, appended with the year of
publication. If there is a single author, then the first three letters
of the author's last name is used instead. A paper by Sirer in 2006
would be cited as ``[Sir06]'' under this scheme. If there are five
or more authors, the first three initials are appended with a ``+''
sign and the year. For instance, a paper by Aardvark, Dewey, Chethem,
and Howe would be cited as ``[ADCH06]'', but if Aardvark and friends
sign on Elvis as a coauthor, the citation string becomes ``[ADC+06]''.
Finally, last name modifiers (such as ``van'') are preserved in lower
case. A paper by Sirer and van Renesse would be cited as ``[SvR07]''.

\texttt{fullname} produces citations that appear like this ``[Walsh and
Sirer 06]''. The last names appear in full for references authored by
up to two authors. A paper by Dewey, Chethem and Howe would be cited as
``[Dewey et al. 06]''.  Fullname citations are the most readable and
should be used whenever possible.

\section{The References Section\label{sec:references}}

\XTX{} provides many options that enable the user to control the
appearance of the references within a document. This section describes
various options that can be passed to the \texttt{crosstex} tool for
achieving the precise formatting desired.

\subsection{Invoking \XTX{}}

In its simplest invocation, \texttt{crosstex} takes one or more file
names, e.g. \texttt{crosstex \textrm{\textit{file1 file2 ...}}}. Each
file is processed separately. Files are found in a search path containing
the directory containing the file being processed and a central system
directory (e.g. \texttt{/usr/local/crosstex/lib}); this search path can
be extended with the \texttt{--dir} option. Extensions (\texttt{.aux},
\texttt{.xtx}, \texttt{.bib}) will be added if necessary to find the
file. Each output will always appear in the same directory as the
file processed, under the same name but with the extension changed to
\texttt{.bbl}.

If \texttt{crosstex} is invoked as \texttt{xtx2bib} or the
\texttt{--xtx2bib} option is given, the output extension will be
\texttt{.bib}, and the bibliographic information will be back-converted
to plain \BibTeX{}.

If \texttt{crosstex} is invoked as \texttt{bib2xtx} or the
\texttt{--bib2xtx} option is given, the output extension will be
\texttt{.xtx}, and the bibliographic information will be output using
\XTX{}'s advanced features where possible.  Currently, it is possible
to use the \texttt{--heading} and \texttt{--reverse-heading} as usual
to specify any number of fields to pull out with \texttt{@default}
statements.  This feature can be very convenient for converting old
\BibTeX{} databases to \XTX{}, but might lose some information if used
on an already optimized \XTX{} database.

If \texttt{crosstex} is invoked as \texttt{xtx2html} or the
\texttt{--xtx2html} option is given, the output extension will be
\texttt{.html}; some style information will be changed as appropriate
for formatting a web bibliography, and the output will be wrapped
into a \LaTeX{} document and translated into HTML by piping it through
\texttt{hevea}. Sometimes it is necessary to run this more than once
to get labels right, as with \LaTeX{}; \texttt{hevea} will print an
appropriate message if this is necessary. By default, the style used
for HTML is pretty non-traditional, but can be overridden by further
options: \texttt{xtx2html --style plain \textrm{\textit{file}}} looks
nice and tame.

A number of options can be specified to change the style of the
bibliography to override or tweak that specified by a document.

\subsection{Abbreviation}

The \texttt{--short} option allows many kinds of objects to be abbreviated
in the bibliography. For example, to use shortened month names (`Jan.',
`Feb.') instead of long ones (`January', `February'), simply use
the option \texttt{--short month}. This allows the creators of the
database to specify both month names just once, refer to the relevant
\texttt{month} objects in their entries, and the formatting of month
names to be consistently chosen when the bibliography is formatted.

Anything with a name can be abbreviated this way---so a conference can
be shortened from ``Networked Systems Design and Implementation'' to
``NSDI'' when under the space crunch or filled back out later with a
simple option. Databases mention each name only once, and, even more
importantly, what name to use is left to the document and the user and
is not imposed on the database maintainer.

Objects that can be shortened include \texttt{author},
\texttt{conference}, \texttt{conferencetrack}, \texttt{country},
\texttt{journal}, \texttt{month}, \texttt{state}, \texttt{string},
and \texttt{workshop}.

\subsection{Authors}

Author names can be complicated, and are the source of much confusion
in \BibTeX{}. The same author might appear with a middle name, without a
middle name, last name first, with abbreviated first names, mis-spelled,
with different combinations of accents, and so forth.

In \XTX{}, the database maintainer can enter the name just once in an
author object and control the way it is formatted via options. The
\texttt{--short author} option generates abbreviated author names
automatically if an author doesn't have an explicitly mentioned short
name, and \XTX{} is careful to handle complicated names with accents
and modifiers correctly when abbreviating or generating citation keys.

The option \texttt{--last-first} causes the first author in each list
to be formatted `Last, First' instead of `First Last'.  \XTX{} does the
Right Thing with modifiers here, too.  When author names are capitalized
with \texttt{--capitalize author}, \XTX{} carefully works around \LaTeX{}
commands and accents to produce clean-looking names.

\subsection{Capitalization}

Any object that can be abbreviated with \texttt{--short} can be coerced to
all upper case with \texttt{--capitalize}.  For example, to cause authors
to appear capitalized, issue \texttt{--capitalize author}.

\subsection{Titles}

Title case is one of the most common inconsistencies when using
\BibTeX{}. Often, some papers are cited with lower-case titles, some are all
upper-case, and some follow mixed title-case. Key acronyms (e.g. BGP)
and proper nouns (e.g. Internet) are haphazardly capitalized, or not,
depending on how diligent the author was when putting together the
bibliographic database.

\XTX{} ensures that all titles follow the same uniform capitalization
standard, even if they appear in a wild variety of styles in the database.
The first letter of each word will become capitalized, the rest lower,
the standard known as ``titlecase''.  \XTX{} is very careful to ensure
the titles come out looking ``good''---words in StudlyCaps or ALLCAPS are
retained as-is, \LaTeX{} commands and anything in math mode are protected,
compound words such as ``Peer-to-Peer'' are split into words, capitalized
correctly, and re-assembled, and additionally a list of known phrases are
carefully found and formatted.  For example, any appearance of a string
that is (ignoring case) equivalent to ``USENIX'' appears as ``USENIX''.
These phrases are found at run-time by \XTX{} in \texttt{@titlephrase}
commands, such as:

\begin{small}\begin{verbatim}
@titlephrase "USENIX"
@titlephrase "Linux"
\end{verbatim}\end{small}

The standard include files define certain common Computer Science phrases
such as these, but they can appear anywhere in the \texttt{.xtx} file.
Small words, such as ``a'', ``an'', ``the'', etc. are also handled
specially: They are made lower-case except at the beginning of the title
or after certain punctuation, such as long dashes or colons.  These,
too, are defined at run-time by \texttt{@titlesmall} commands:

\begin{small}\begin{verbatim}
@titlesmall "a"
@titlesmall "the"
\end{verbatim}\end{small}

Again, the standard include files define important English small words
to start with.

An example title with the default might be ``Aardvark: A System for
Peer-to-Peer BGP Routing on the Internet''.

With \texttt{--titlecase lower}, Only the first letter of the title and
those following punctuation are capitalized, the rest put into lower-case.
All of the special cases for the default title-case still apply.  Thus,
the example title would appear ``Aardvark: A system for peer-to-peer
BGP routing on the Internet''.

With \texttt{--titlecase upper}, everything, even known phrases and
small words, are put into upper-case thus: ``AARDVARK: A SYSTEM FOR
PEER-TO-PEER BGP ROUTING ON THE INTERNET''.  Commands and math-mode are
still protected.

Finally, to allow titles to appear as they are specified in the database,
use \texttt{--titlecase as-is}.

\subsection{Proceedings}

There are a variety of styles in use when citing papers at
conferences. Some people prefer to precede the conference name with
``In Proceedings of the ''.  These same people usually use ``In Proc. of
'' when pressed for space.  With \texttt{--add-proceedings}, \XTX{} will
generate book titles for conferences beginning with ``In Proceedings of'',
while \texttt{--add-proc} uses the shorter ``In Proc.'' and without any
options, only ``In '' is used for papers in conferences with proceedings.

For journal articles, the usual convention is to simply put the journal
name in italics following the author names, and this is the default \XTX{}
and \BibTeX{} behavior. Some people prefer to prepend ``In '' to the
name of the journal; this can be accomplished with the \texttt{--add-in}
option.

\BibTeX{} users affect these personal preferences by modifying the
bibliographic database. Such changes are potentially disruptive and can
introduce errors.  \XTX{} enables such stylistic changes, which do not
affect the underlying data, to be affected without modifying the database,
and ensures that the choice will be applied consistently throughout.

\subsection{Sorting and Headings}

Sorting affects the order in which references will appear in the
bibliography.  By default, entries will be sorted by their citation keys,
or by their authors and publication dates, depending on the citation
style.  The \texttt{--sort} option provides finer control over the
sort order. By specifying \texttt{--sort \textrm{\textit{field}}}, the
database will be stably sorted by field; later specifying \texttt{--sort
\textrm{\textit{field2}}} will cause the bibliography to be sorted by
field2, but the entries will still be sub-sorted by the first field. To
sort in descending order, use \texttt{--reverse-sort} in the same way.

When processing large bibliographies, it can be nice to partition
the entries into labeled categories. Specifying the \texttt{--heading
\textrm{\textit{field}}} option specifies a field to be used to divide
the entries into sections. For example, \texttt{--heading year} will
cause the entries to be grouped by year and given headings for each
different year. (\texttt{--reverse-heading} will reverse the order in
which the sections appear.) When converting a bibliography of personal
publications to HTML, for example, it might be convenient to group by an
information field such as \texttt{--heading category} to nicely organize
the produced bibliography.

\subsection{Hyperlinks}

\XTX{} supports searching fields to find hyperlinks and presenting them
in the references section.  This is useful for any target format with
hyperlinks, including PDF and HTML.  Normally, no fields are treated
as possible links, except when converting to HTML, when the list
defaults to Abstract, URL, PS, PDF, HTML, DVI, TEX, BIB, FTP, HTTP,
and RTF.  A new field can be added with \texttt{-l~}\textit{field} or
\texttt{--link~}\textit{field}; \texttt{--no-link} clears the list in
order to disable link-finding or start over.

If any of the fields, case insensitive (e.g. \texttt{--link~PDF} and
\texttt{--link~pdf} are equivalent), consists of a URL, it will appear
at the end of the reference as a hyperlink with its label as the name
of the field given to \texttt{--link} (e.g. the former would match the
same field, but produce links labeled ``PDF'' and ``pdf'' respectively).

\subsection{Abstracts and Keywords}

Some detailed database entries might include a list of keywords related
to the paper or even a complete abstract.  By default, these fields are
accepted but do not appear in the reference.  With \texttt{--abstract},
abstracts will appear in blocks following the appropriate entries.
The \texttt{--keywords} option invokes similar behavior for keyword
lists.  To explicitly set the defaults, \texttt{--no-abstract} and
\texttt{--no-keywords} disable these extra blocks.

If both \texttt{--link~Abstract} and \texttt{--abstract} are specified,
\XTX{} tries to be intelligent: The abstract will either appear as a
link or as a block, based on whether it appears to be a complete text
or a link to the real abstract.

\subsection{Putting the title first}

Ordinarily, each entry begins with the author or editor first, then the
title.  A simple kind of re-ordering can be accomplished by specifying
\texttt{--title-head}, which causes the title to come first and bold.
This option can be negated with \texttt{--no-title-head}, but it is
default only with \texttt{--style~html} or \texttt{--xtx2html}.

\subsection{Splitting up lines}

Ordinarily, each entry takes up one logical line, which might wrap.
The \texttt{--break-lines} option instead puts each major field (author,
title, publication information, any hyperlinks, etc.) on its own line,
in the same order they would have appeared on a single line.  This can be
combined with putting the title first with \texttt{--title-head} to cause
the title to come on the first line all by itself, which is the default
with \texttt{--style~html} or \texttt{--xtx2html}.  To explicitly cause
entries to appear on a single logical line, use \texttt{--no-break-lines}.

\subsection{Label appearance}

By default, each entry in the references section is labeled with
its citation key as it would appear in the document, e.g. ``[1]'' or
``[WS04]''.  It is also possible, with the option \texttt{--blank-labels},
to leave entries un-labeled; this does not influence how citations
appear in the document body in any way, but leaves out labels in the
bibliography.

By itself, this option would probably produce a bibliography in which it
is impossible to track down citations.  However, it can be useful when
converting a database to HTML, for example, when there is no document
anyway and the labels look messy.  Thus, leaving labels blank is the
default with \texttt{--style~html} or \texttt{--xtx2html}.  To explicitly
include labels, use \texttt{--no-blank-labels}.

\subsection{Styles}

Styles, such as \texttt{plain}, \texttt{abbrv}, etc., are simply
collections of arguments to \XTX{} that produce the appropriate
appearance, and \XTX{} is designed to make it easy to change or add
styles.  The style \texttt{html} sets a bunch of link fields and other
stylistic choices appropriate for the web, and the basic \BibTeX{}
styles are also implemented by default:

\begin{description} \item[\texttt{abbrv}] is equivalent to
  \texttt{--no-break-lines --no-blank-labels --no-title-head --no-abstract
  --no-keywords --short author --cite-by initials}
\item[\texttt{alpha}] is equivalent to
  \texttt{--no-break-lines --no-blank-labels --no-title-head --no-abstract
  --no-keywords --cite-by initials}
\item[\texttt{full}] is equivalent to
  \texttt{--no-break-lines --no-blank-labels --no-title-head --no-abstract
  --no-keywords --cite-by fullname --add-proceedings --add-in}
\item[\texttt{unsrt}] is equivalent to
  \texttt{--no-break-lines --no-blank-labels --no-title-head --no-abstract
  --no-keywords --cite-by number --no-sort}
\item[\texttt{plain}] is equivalent to
  \texttt{--no-break-lines --no-blank-labels --no-title-head --no-abstract
  --no-keywords --cite-by number}
\item[\texttt{html}] is equivalent to
  \texttt{-l Abstract -l URL -l PS -l PDF -l HTML -l DVI -l TeX -l
  Bib -l FTP -l HTTP -l RTF --break-lines --blank-labels --title-head
  --abstract --keywords}
\end{description}

Styles allow complex requirements to be conveniently bundled, either
in the \verb"\bibliographystyle" command or on the command line with
\texttt{--style~}\textit{name}.

\subsection{Inside the \texttt{.tex} file}

The \texttt{.tex} source file never needs to know you're using
\XTX{}, because it is completely backwards compatible with normal
\LaTeX{} auxiliary files that note citations and styles. The
\verb"\bibliographystyle{"\textit{foo}\verb"}" command will
cause the equivalent of the command-line option \texttt{--style
\textrm{\textit{foo}}}.

However, the \verb"\bibliographystyle" command can cause other magic as
well. Arbitrary command line options may be specified after the style
name, separated by single spaces. For example:

\begin{small}\begin{verbatim}
\bibliographystyle{plain --add-in --add-proceedings --short author}
\end{verbatim}\end{small}

This allows documents to have fine-grained control over styling. Run-time
options will still take precedence over document defaults.

\subsection{Adding citations}

Normally, only entries that are cited in the \LaTeX{} document appear
in the references section.  When processing a database directly,
the default is for all entries to appear.  These behaviors can be
manipulated at run-time with the \texttt{--cite} option.  It behaves
exactly as a \LaTeX{} \verb"\cite" command, including the use of the
asterisk (\texttt{--cite *}, modulo shell-escaping) to cite all entries,
and constrained citation.  When processing \LaTeX{} documents, this
adds entries to the references; when processing databases directly,
this overrides the default to cite everything and cites only the entries
specifically mentioned on the command line.

\subsection{Manipulating \texttt{crosstex}}

Numerous options help control or debug \texttt{crosstex} itself.  The
\texttt{--version} option will cause \texttt{crosstex} to print out its
version and exit.  To get a list of all supported options and brief
descriptions, use \texttt{--help} or \texttt{-h}.

Two options change the level of error reporting.  By default, only
errors that will definitely change the appearance of the bibliography
are produced.  With \texttt{--strict}, more warnings, such as for unknown
fields or other problems, will be printed.  With \texttt{--quiet}, on the
other hand, no errors or warnings will appear at all.

The \texttt{--dump} option provides very detailed debugging.  Processing
will continue exactly as normal, but at the end, any kinds of objects
specified to \texttt{--dump} will be listed as output.  For example, to
list all the \texttt{author} objects defined in a bibliography \textit{foo},
\texttt{crosstex --dump author foo} would process foo and print a list.

Two additional, non-object dumps are permitted.  With \texttt{--dump~file},
\texttt{crosstex} will print a trace of the path to every database it
processes; this allows one to examine whether it is choosing the right files
based on the search path and explore what standard databases are being
automatically pulled in.  The various title phrases and small words that
control title capitalization can be dumped with \texttt{--dump~titlephrase}.


\section{Extending \XTX{}\label{sec:extending}}

\XTX{} is designed to be easy to extend with very trivial knowledge
of Python. Before continuing, it is very important to have a look at
the standard object types and fields in Appendix~\ref{sec:standard},
which are already supported by \XTX{}. New objects or fields are defined
by editing the file \texttt{crosstexobjects.py}, which is typically
installed in \texttt{/usr/local/crosstex/lib}.

To create a new field for a particular object type, find its definition
(e.g., the section defining the \texttt{string} object begins
\texttt{class string}). Most objects already define some fields;
simply copy that syntax for your own field.

Fields are defined as optional or required by assigning them the
values \texttt{OPTIONAL} and \texttt{REQUIRED}, respectively.  To make
an optional field required or a required field optional, simply assign
it the new value in the class where you want the change.

To create an entirely new class \textit{foo} which is identical to
a current one named \textit{bar}, add the following to the end of
\texttt{crosstexobjects.py}.

\begin{small}\begin{verbatim}
class foo(bar):
    pass
\end{verbatim}\end{small}

If \textit{foo} should have different required or optional fields form
\textit{bar}, define them as noted above. (The \texttt{pass} is only
necessary if the body of the \texttt{class} definition is empty, e.g. it
does not set the values of any fields.)

If you are more advanced at Python programming and want to change
the behavior of objects, simply look at the ones that exist. The
string value of an object is obtained, appropriately enough, from its
\texttt{\_\_str\_\_} method, should you want to change the appearance of
any objects. Similarly, every object has access to an \texttt{\_options}
member containing the options set on the command line.

Happy hacking!


\appendix
\section{Standard object types\label{sec:standard}}

These are the kinds of objects \XTX{} knows about by default. For
information about extending this notion, see Extending \XTX{} in
Section~\ref{sec:extending}.

\begin{description}

\item[\texttt{string}]
\textsc{Required:} \texttt{name} and/or \texttt{shortname}
(\texttt{longname} is an alias for \texttt{name}.)
\textsc{Relevant arguments:} \texttt{--short}

\item[\texttt{author}] As \texttt{string}, except:
\textsc{Optional:} \texttt{address}, \texttt{affiliation}, \texttt{email},
\texttt{institution}, \texttt{organization}, \texttt{phone},
\texttt{school}, \texttt{url}

\item[\texttt{state}] As \texttt{string}, except:
\textsc{Optional:} \texttt{country}

\item[\texttt{country}] As \texttt{string}.

\item[\texttt{location}]
\textsc{Optional:} \texttt{city}, \texttt{state}, \texttt{country}

\item[\texttt{month}] As \texttt{string}.

\item[\texttt{journal}] As \texttt{string}.

\item[\texttt{newspaper}] As \texttt{journal}.

\item[\texttt{misc}]
\textsc{Optional:}
\texttt{abstract},
\texttt{address},
\texttt{affiliation},
\texttt{annote},
\texttt{author},
\texttt{bib},
\texttt{bibsource},
\texttt{booktitle},
\texttt{category},
\texttt{chapter},
\texttt{contents},
\texttt{copyright},
\texttt{crossref},
\texttt{doi},
\texttt{dvi},
\texttt{edition},
\texttt{editor},
\texttt{ee},
\texttt{ftp},
\texttt{howpublished},
\texttt{html},
\texttt{http},
\texttt{institution},
\texttt{isbn},
\texttt{issn},
\texttt{journal},
\texttt{key},
\texttt{keywords},
\texttt{language},
\texttt{lccn},
\texttt{location},
\texttt{month},
\texttt{monthno},
\texttt{mrnumber},
\texttt{note},
\texttt{number},
\texttt{organization},
\texttt{pages},
\texttt{pdf},
\texttt{price},
\texttt{ps},
\texttt{publisher},
\texttt{rtf},
\texttt{school},
\texttt{series},
\texttt{size},
\texttt{title},
\texttt{type},
\texttt{url},
\texttt{volume},
\texttt{year}
\textsc{Relevant arguments:} \texttt{--cite-by}, \texttt{--titlecase},
\texttt{--link}, \texttt{--abstract}, \texttt{--keywords}

\item[\texttt{article}] As \texttt{misc}, except:
\textsc{Required:} \texttt{author}, \texttt{title}, \texttt{journal},
\texttt{year}
\textsc{Relevant arguments:} \texttt{--add-in}

\item[\texttt{newspaperarticle}] As \texttt{article}, except:
(\texttt{newspaper} is an alias for \texttt{journal})

\item[\texttt{book}] As \texttt{misc}, except:
\textsc{Required:} \texttt{author} and/or \texttt{editor}, \texttt{title},
\texttt{publisher}, \texttt{year}

\item[\texttt{booklet}] As \texttt{misc}, except:
\textsc{Required:} \texttt{title}

\item[\texttt{inbook}] As \texttt{misc}, except:
\textsc{Required:} \texttt{author} and/or \texttt{editor}, \texttt{title},
\texttt{chapter} and/or \texttt{pages}, \texttt{publisher}, \texttt{year}

\item[\texttt{incollection}] As \texttt{misc}, except:
\textsc{Required:} \texttt{author}, \texttt{title}, \texttt{booktitle},
\texttt{publisher}, \texttt{year}

\item[\texttt{inproceedings}] As \texttt{misc}, except:
\textsc{Required:} \texttt{author}, \texttt{title}, \texttt{booktitle},
\texttt{year}
\textsc{Relevant arguments:} \texttt{--add-proceedings}, \texttt{--add-proc}

\item[\texttt{manual}] As \texttt{misc}, except:
\textsc{Required:} \texttt{title}

\item[\texttt{thesis}] As \texttt{misc}, except:
\textsc{Required:} \texttt{author}, \texttt{title}, \texttt{school},
\texttt{year}

\item[\texttt{mastersthesis}] As \texttt{thesis}.

\item[\texttt{phdthesis}] As \texttt{thesis}.

\item[\texttt{patent}] As \texttt{misc}, except:
\textsc{Required:} \texttt{author}, \texttt{title}, \texttt{number},
\texttt{month}, \texttt{year}

\item[\texttt{proceedings}] As \texttt{misc}, except:
\textsc{Required:} \texttt{title}, \texttt{year}

\item[\texttt{collection}] As \texttt{proceedings}.

\item[\texttt{techreport}] As \texttt{misc}, except:
\textsc{Required:} \texttt{author}, \texttt{title}, \texttt{institution},
\texttt{year}

\item[\texttt{unpublished}] As \texttt{misc}, except:
\textsc{Required:} \texttt{author}, \texttt{title}, \texttt{note}

\item[\texttt{conference}] As \texttt{string}, except:
\textsc{Optional:}
\texttt{address},
\texttt{crossref},
\texttt{editor},
\texttt{institution},
\texttt{isbn},
\texttt{key},
\texttt{keywords},
\texttt{language},
\texttt{location},
\texttt{month},
\texttt{publisher},
\texttt{url},
\texttt{year}

\item[\texttt{conferencetrack}] As \texttt{conference}, except:
\textsc{Optional:} \texttt{conference}

\item[\texttt{workshop}] As \texttt{conferencetrack}.

\item[\texttt{rfc}] As \texttt{misc}, except:
\textsc{Required:} \texttt{author}, \texttt{title}, \texttt{number},
\texttt{month}, \texttt{year}

\item[\texttt{url}] As \texttt{misc}, except:
\textsc{Required:} \texttt{url}
\textsc{Optional:} \texttt{accessyear}, \texttt{accessmonth}

\end{description}

\end{document}
